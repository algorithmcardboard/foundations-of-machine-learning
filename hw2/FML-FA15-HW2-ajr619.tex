\documentclass{article}
\usepackage{lmodern}
\usepackage{amsmath}
\usepackage{amssymb}
\usepackage[T1]{fontenc}
\usepackage{fancyhdr}
\pagestyle{fancy}
\lhead{Anirudhan J Rajagopalan - ajr619 - N18824115} % chktex 8

\begin{document}

\title{Foundations of Machine Learning --- Homework Assignment 1}
\date{October 11, 2015}
\author{Anirudhan J Rajagopalan\\ N18824115\\ ajr619}

\maketitle

\newpage

\section*{D. Kernels}
\subsection*{1}
\begin{description}
  \item{Given:} Kernel, K is defined by \(K(x,y) = \sum_{i=1}^{N} \cos^{n} (x_{i}^{2} - y_{i}^{2} )\) for all \((X, Y) \in \mathbb{R}^{N} \times \mathbb{R}^{N} \) 
  \item{Solution:}  We know that
    \begin{equation}
      \cos (x_{i}^{2} - y_{i}^{2}) = \sin (x_{i}^{2}).\sin (y_{i}^{2}) + \cos (x_{i}^{2}).\cos (y_{i}^{2})
    \end{equation}
    This can be written as a dot product of two vectors 
    \begin{align}
    \phi(x_{i}) = \begin{bmatrix} \cos (x_{i}^{2}) \\ \sin (x_{i}^{2}) \end{bmatrix} && \mathrm{and} &&
    \phi(y_{i}) = \begin{bmatrix} \cos (y_{i}^{2}) \\ \sin (y_{i}^{2}) \end{bmatrix}
    \end{align}

    We know that if K can be written as \( \langle \phi(x_{i}), \phi(y_{i}) \rangle \), then it is a PDS@.

    Also, \( \langle \phi(x_{i}), \phi(y_{i}) \rangle \) is a scalar.  When a scalar is raised to a positive power (n in our case) and summed with N other positive scalar, we get a positive scalar as our answer.  Hence
    \begin{equation*}
      K(x,y) = \sum_{i=1}^{N} \cos^{n} (x_{i}^{2} - y_{i}^{2} ) \mathrm{is PDS.}
    \end{equation*}
\end{description}

\end{document}
